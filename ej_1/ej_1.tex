%% LyX 2.1.4 created this file.  For more info, see http://www.lyx.org/.
%% Do not edit unless you really know what you are doing.
\documentclass[english]{article}
\usepackage[T1]{fontenc}
\usepackage[latin9]{inputenc}
\setlength{\parskip}{\smallskipamount}
\setlength{\parindent}{0pt}
\usepackage{amsmath}
\usepackage{babel}
\begin{document}

\title{Ejercicios Sistemas Complejos}


\author{Antonio Jes�s Mart�n Valverde}

\maketitle

\section{Calcula los cumulantes de la distribuci�n de Poisson:}

\[
P_{m}(a)=\dfrac{1}{m!}a^{m}e^{-a}
\]



\section*{con n definida en el rango de los enteros positivos o nulos, es decir,
$n=0,1,2,...$ y siendo $a$ una constante positiva.}

Consideramos la funci�n generadora de cumulantes:

\[
K(t)=\ln<e^{tm}>=\ln\left[\sum_{m=0}^{\infty}\dfrac{1}{m!}a^{m}e^{-a}e^{tm}\right]=\ln\left[e^{-a}\sum_{m=0}^{\infty}\dfrac{1}{m!}\left(a\,e^{t}\right)^{m}\right]=\ln\left[e^{-a}e^{a\,e^{t}}\right]=a(e^{t}-1)
\]


Se tiene que el cumulante $\alpha$-�simo es:

\[
\xi_{\alpha}=\left.\dfrac{d^{\alpha}K(t)}{dt^{\alpha}}\right|_{t=0}
\]


En nuestro caso se tiene que:

\[
\left.\dfrac{d^{\alpha}K(t)}{dt^{\alpha}}\right|_{t=0}=a\,e^{t}\,\forall\alpha\geq1\,\Longrightarrow\xi_{\alpha}=a\,\forall\alpha\geq1
\]



\section{Calcula las relaciones entre cumulantes y momentos hasta orden 4.}


\section{Estudiar anal�ticamente y por medio del ordenador, la serie temporal
resultante de iterar el mapa de la figura. En particular, calcular
anal�ticamente el valor del exponente de Lyapunov y mostrar que es
positivo. �Qu� implicaciones tiene este hecho? Dibujar un gr�fico
con $x_{i}$ en el eje de abscisas y $x_{i+1}$ en el eje de ordenadas.
�Qu� aspecto tiene? Verificar que aparece la forma ilustrada abajo
(esto es que hay un desplazamiento de los decimales).}
\end{document}
